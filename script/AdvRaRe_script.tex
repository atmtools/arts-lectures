\documentclass[a4paper,fleqn]{article}
%% Seitenaufbau
\usepackage[top=3cm, bottom=2.5cm, left=3.5cm, right=3.5cm]{geometry}

%% Schriftbild
\usepackage{lmodern}  % Latin Modern Zeichensatz
\usepackage[utf8]{inputenc}  % Unterstuetzung von Umlauten im Quelltext
\usepackage[T1]{fontenc}  % Korrekte Umlaute im Output
\renewcommand{\familydefault}{\sfdefault}  % Serifenlose Schrift
\usepackage{setspace}\onehalfspacing  % 1.5-facher Zeilenabstand
\renewcommand{\arraystretch}{1.5}  % 1.5-facher Zeilenabstand (Tabellen)
\setlength{\parindent}{0pt}  % Keine Einrueckung am Beginng von Absaetzen
\usepackage{fancyhdr}
\pagestyle{fancy}
\renewcommand{\headrulewidth}{0.5pt}
\renewcommand{\sectionmark}[1]{\markright{\thesection\ #1}}
\lhead{\rightmark}
\chead{}
\cfoot{\thepage}  % Richtige Schriftart fuer Seitenzahlen
\sloppy  % Weniger strikte Silbentrennung

%% Verlinkung von Inhaltsverzeichnis, Bildern und Formeln
\usepackage[pagebackref]{hyperref}  % Verlinkung von URLs und Referenzen
\usepackage{color}  % Definition von Linkfarben
\definecolor{DarkRed}{rgb}{0.5,0,0}
\hypersetup{
  colorlinks,
  citecolor=DarkRed,
  linkcolor=DarkRed,
  urlcolor=blue}

%% Mathematikumgebung
\usepackage{mathtools}
\usepackage{amssymb}  % Erweiterte Bibliothek mathematischer Symbole
\usepackage{euler}  % Serifenlose Schrift in Formelumgebungen
\renewcommand{\epsilon}{\varepsilon}  % Nutze "richtiges" Epsilon

%% Grafikumgebungen
\usepackage{graphicx}  % Erweiterte Grafikumgebung
\usepackage{float}  % Automatische Positionierung von Bildern
\usepackage{floatflt}  % Grafiken im Text einbetten
\usepackage{subcaption}  %  Bildunterschriften fuer subfigures

%% Listenumgebung
\usepackage{enumerate}
\usepackage{textcomp}  % Korrekte serifenlose Aufzählungszeichen

%% Farbige Umrahmungen
\usepackage{framed}
\definecolor{shadecolor}{rgb}{0.9,0.9,0.9}

%% Inhalt der Titelseite
\title{Advanced radiation and remote sensing}
\author{Stefan A Buehler, Lukas Kluft, FIXME all that worked on
the script}
\date{\today}

%%%%%%%%%%%%%%%%%%%%%%%%%%%%%%%%%%%%%%%%%%%%%%%%%%%%%%%%%%%%%%%%%%%%%%
\begin{document}
\maketitle
\thispagestyle{empty}\pagestyle{empty}
\tableofcontents
\newpage\pagestyle{fancy}

\section{Introduction}

FIXME

\section{Rotational spectra}

\subsection{Energy states and photons}
Absorption and emission of a photon can happen if the internal energy of a
molecule changes. The difference in two energiy states $E$ and the freqency of
the photon $\nu$ are connected through the Planck constant $h$.
\begin{equation}
  E_f - E_i = \Delta E = h \nu
\end{equation}
The transition between different energy levels can be divided into three types,
corresponding to different states of the molecule.  Each transition type is
characteristic for a specific region of the electro-magnetic spectra.  One can
distinguish changes in a molcules rotation (microwave), vibration (infrared)
and electronic transitions (UV/visible).

Molecules consist of different atoms. There are diatomic molecules with two
atoms ($O_2$, $N_2$, $NO$, $HCl$, $CO$) and polyatomic molecules with
three or more atoms ($O_3$, $NO_2$, $CO_2$). We will mainly discuss
diatomic molecules, because they are easier to understand.

One can distinguish between three types of interaction between a photon and a
molecule: absorption, spontaneous emission and stimulated emission.  There are
three condtions to be fulfilled for a molecule to absorb a photon:
\begin{enumerate}
\item The frequency of the photon has to match one the molecule's energy
transitions.
\begin{equation}
  \Delta E = h \nu
\end{equation}

\item The net absorption has to be proportional to the difference of occupation
probabilities of states $i$ and $f$: $p_i - p_f$. Where $p_i$ is given by the
Boltzmann distribution (in local thermodynamic equilibrium).
\begin{equation}
  p_i = \frac{q_i e^{-E_i / (kT)}}
        {\sum_{j=1}^\infty q_j e^{-E_j / (kT)}}
\end{equation}
The denominator is often referred to as $Q(T)$. It has various names like total
internal partition sum, partition function, or Zustandssumme.

\item Absorption is proportional to the electric dipole matrix element $\mu_{if}$
between states $i$ and $f$.
\begin{equation}
  \mu_{if} = \mu \iiint \psi_f^*(x,y,z) \psi(x,y,z) \,dx\,dy\,dz
\end{equation}
Where $\mu$ is the magnitude of the dipole operator. $\mu_{if}$ is a constant
for each transition.
\end{enumerate}

Combining all conditions, one can define the absorption coefficient $\alpha$.
\begin{equation}
  \alpha(\nu) = n \sum_{f,i} S_{fi}(T) \delta(\nu_{fi} - \nu)
\end{equation}
With number density $n$, line strength $S_{fi}$, difference of state energies
$h\nu_{fi}$ and Dirac's delta function $\delta$ (physical not true, will be
replaced later by a so called "line shape" function.

The line strength is different for each transition and defined as follows.
\begin{equation}
    S_{fi} = \frac{8\pi^3\nu_{fi} \lvert\mu_{fi}\rvert^2 q_i}{3hcQ(T)}
      \left( e^{E_i/(kT)} - e^{E_f/(kT)} \right)
\end{equation}
Although this equations looks complicated, there is only one true independent
variable, the temperature $T$. All other variables are constants of the
transition, or fundamental constants.

Spectral line catalogues are collections of these transition parameters. There
is one line in the catalogue for each spectral line (thousands for each
molecule). The most well-known is the HITRAN catalogue.

\subsection{Rotational transitions}

\section{Vibrational spectra}

\section{Line shape}

\section{Thermal radiation}

FIXME: Mostly covered by script from Bachelor lecture, but at least
state the Planck law here.

\section{Radiative transfer equation}

FIXME: Mostly covered by script from Bachelor lecture, but at least
state the RTE here.

\section{Jacobians}

\end{document}
