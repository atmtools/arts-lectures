\documentclass[a4paper,fleqn]{article}
%% Seitenaufbau
\usepackage[top=3cm, bottom=2.5cm, left=3.5cm, right=3.5cm]{geometry}

%% Schriftbild
\usepackage{lmodern}  % Latin Modern Zeichensatz
\usepackage[utf8]{inputenc}  % Unterstuetzung von Umlauten im Quelltext
\usepackage[T1]{fontenc}  % Korrekte Umlaute im Output
\renewcommand{\familydefault}{\sfdefault}  % Serifenlose Schrift
\usepackage{setspace}\onehalfspacing  % 1.5-facher Zeilenabstand
\renewcommand{\arraystretch}{1.5}  % 1.5-facher Zeilenabstand (Tabellen)
\setlength{\parindent}{0pt}  % Keine Einrueckung am Beginng von Absaetzen
\usepackage{fancyhdr}
\pagestyle{fancy}
\renewcommand{\headrulewidth}{0.5pt}
\renewcommand{\sectionmark}[1]{\markright{\thesection\ #1}}
\lhead{\rightmark}
\chead{}
\cfoot{\thepage}  % Richtige Schriftart fuer Seitenzahlen
\sloppy  % Weniger strikte Silbentrennung

%% Verlinkung von Inhaltsverzeichnis, Bildern und Formeln
\usepackage[pagebackref]{hyperref}  % Verlinkung von URLs und Referenzen
\usepackage{color}  % Definition von Linkfarben
\definecolor{DarkRed}{rgb}{0.5,0,0}
\hypersetup{
  colorlinks,
  citecolor=DarkRed,
  linkcolor=DarkRed,
  urlcolor=blue}

%% Mathematikumgebung
\usepackage{mathtools}
\usepackage{amssymb}  % Erweiterte Bibliothek mathematischer Symbole
\usepackage{euler}  % Serifenlose Schrift in Formelumgebungen
\renewcommand{\epsilon}{\varepsilon}  % Nutze "richtiges" Epsilon

%% Grafikumgebungen
\usepackage{graphicx}  % Erweiterte Grafikumgebung
\usepackage{float}  % Automatische Positionierung von Bildern
\usepackage{floatflt}  % Grafiken im Text einbetten
\usepackage{subcaption}  %  Bildunterschriften fuer subfigures

%% Listenumgebung
\usepackage{enumerate}
\usepackage{textcomp}  % Korrekte serifenlose Aufzählungszeichen

%% Farbige Umrahmungen
\usepackage{framed}
\definecolor{shadecolor}{rgb}{0.9,0.9,0.9}

%% Chemische Formeln
\usepackage{chemfig}

%% Inhalt der Titelseite
\title{Advanced radiation and remote sensing}
\author{Stefan A Buehler, Lukas Kluft, Christoph Sauter, Jakob Doerr, FIXME all that worked on
the script}
\date{\today}

%%%%%%%%%%%%%%%%%%%%%%%%%%%%%%%%%%%%%%%%%%%%%%%%%%%%%%%%%%%%%%%%%%%%%%
\begin{document}
\maketitle
\thispagestyle{empty}\pagestyle{empty}
\tableofcontents
\newpage\pagestyle{fancy}

% CHAPTER 1 - INTRODUCTION
\section{Introduction}

FIXME

% CHAPTER 2 ROTATIONAL SPECTRA
\section{Rotational spectra}

\subsection{Energy states and photons}
Absorption and emission of a photon can happen if the internal energy of a
molecule changes. The difference in two energy states $E$ and the frequency of
the photon $\nu$ are connected through the Planck constant $h$.
\begin{equation}
  E_f - E_i = \Delta E = h \nu
\end{equation}
The transition between different energy levels can be divided into three types,
corresponding to different states of the molecule.  Each transition type is
characteristic for a specific region of the electro-magnetic spectra.  One can
distinguish changes in a molecules rotation (microwave), vibration (infrared)
and electronic transitions (UV/visible).

Molecules consist of different atoms. There are diatomic molecules with two
atoms ($O_2$, $N_2$, $NO$, $HCl$, $CO$) and polyatomic molecules with
three or more atoms ($O_3$, $NO_2$, $CO_2$). We will mainly discuss
diatomic molecules, because they are easier to understand.

One can distinguish between three types of interaction between a photon and a
molecule: absorption, spontaneous emission and stimulated emission.  There are
three conditions to be fulfilled for a molecule to absorb a photon:
\begin{enumerate}
\item The frequency of the photon has to match one the molecule's energy
transitions.
\begin{equation}
  \Delta E = h \nu
\end{equation}

\item The net absorption has to be proportional to the difference of occupation
probabilities of states $i$ and $f$: $p_i - p_f$. Where $p_i$ is given by the
Boltzmann distribution (in local thermodynamic equilibrium).
\begin{equation}
  p_i = \frac{q_i e^{-E_i / (kT)}}
        {\sum_{j=1}^\infty q_j e^{-E_j / (kT)}}
\end{equation}
The denominator is often referred to as $Q(T)$. It has various names like total
internal partition sum, partition function, or Zustandssumme.

\item Absorption is proportional to the electric dipole matrix element $\mu_{if}$
between states $i$ and $f$.
\begin{equation}
  \mu_{if} = \mu \iiint \psi_f^*(x,y,z) \psi(x,y,z) \,dx\,dy\,dz
\end{equation}
Where $\mu$ is the magnitude of the dipole operator. $\mu_{if}$ is a constant
for each transition.
\end{enumerate}

Combining all conditions, one can define the absorption coefficient $\alpha$.
\begin{equation}
  \alpha(\nu) = n \sum_{f,i} S_{fi}(T) \delta(\nu_{fi} - \nu)
\end{equation}
With number density $n$, line strength $S_{fi}$, difference of state energies
$h\nu_{fi}$ and Dirac's delta function $\delta$ (physical not true, will be
replaced later by a so called "line shape" function.

The line strength is different for each transition and defined as follows.
\begin{equation}
    S_{fi} = \frac{8\pi^3\nu_{fi} \lvert\mu_{fi}\rvert^2 q_i}{3hcQ(T)}
      \left( e^{E_i/(kT)} - e^{E_f/(kT)} \right)
\end{equation}
Although this equations looks complicated, there is only one true independent
variable, the temperature $T$. All other variables are constants of the
transition, or fundamental constants.

Spectral line catalogues are collections of these transition parameters. There
is one line in the catalogue for each spectral line (thousands for each
molecule). The most well-known is the HITRAN catalogue.


\subsection{Rotational transitions}
Every free molecule has 3 \textbf{Moments of inertia} $I_A$, $I_B$, $I_C$ (\textit{Trägheitsmomente}), since there are 3 principal axes of rotation in the x-y-z-plane. We can use the Moments of inertia to classify the molecules and devide them into groups:


\begin{center}
\begin{tabular}{| c | c | c | c |}
  \hline
  Linear & Spherical top & Symmetric top & Asymmetric top \\
  \hline
  & & &  \\
\chemfig{H-[,0.8]Cl} &
\chemfig{C(-[:330]H)(-[:90,0.8]H)(-[:210]H)(-[:270,0.8]H)} &
\chemfig{C(-[:0,0.8]F)(-[:140]H)(-[:180,0.8]H)(-[:220]H)} &
\chemfig{H-[:30,0.8]O-[:-30,0.8]H} \\
 & & & \\
  %\hline
  $I_A \approx 0$,   $I_B = I_C$ &
  $I_A = I_B = I_C$ &
  $I_A \neq 0$,   $I_B = I_C \neq I_A$ & 
  $I_A \neq I_B \neq I_C$ \\
  \hline
\end{tabular}
\end{center}

From classical mechanics, we can define the Rotational energy
\begin{equation}
E_r = \frac{1}{2} I \omega^2 = \frac{J^2}{2I},
\end{equation}
where $\omega$ is the angular frequency (\textit{Kreisfrequenz}), $I$ the moment if inertia and $J = I\cdot\omega$ the angular momentum (\textit{Drehmoment}).\\
In case of a diatomic molecule, we can use the \textbf{reduced mass $\mu$} to express some properties of the molecule more easily. The relative motion of such a diatomic molecule is the same as that of a simple particle with the reduced mass $\mu$. It can be calculated by taking into account the two individual masses $m_1, m_2$ of the individual atoms.
\begin{equation}
\mu = \frac{m_1 m_2}{m_1+m_2} \qquad or \qquad \frac{1}{\mu} = \frac{1}{m_1} + \frac{1}{m_2}
\end{equation}
If we devide the distance between the mass centers of the atoms $r_0$ into their individual distances to the axis of rotation $r_1, r_2$, so that $r_0 = r_1 + r_2$, we can define the center of gravity:
\begin{equation*}
m_1 r_1 = m_2 r_2
\end{equation*}
The general formula for the moment of inertia 
\begin{equation*}
I = \sum_i m_i r_i^2 
\end{equation*}
can therefore be expressed as 
\begin{equation}
I = \mu r_0^2.
\end{equation}
Now we look at two extreme cases for the reduced mass:
\begin{enumerate}
\item The masses of both atoms are equal ($m_1 = m_2$) \\
  $\Rightarrow \mu = \frac{m_1^2}{2m_1} = \frac{1}{2}m_1 = \frac{1}{2}m_2$
\item One mass is much larger than the other ($m_1 >> m_2$) \\
  $\Rightarrow \mu = \frac{m_2}{\frac{m_1 + m_2}{m_1}} \approx m_2$ 
\end{enumerate}
In Quantum mechanics, you can express energy levels by using the Hamiltonian function (total energy)
\begin{equation}
\mathcal{H}=\frac{J^2}{2I},
\end{equation}
which leads to a discrete number of allowed energy solutions:
\begin{equation}
E_\mathcal{J} = \frac{\hbar^2}{2I} \mathcal{J}(\mathcal{J}+1), \qquad \text{with} \quad \mathcal{J} = 0, 1, 2, ...
\end{equation}
$\hbar$ = $h/2\pi$ is the Planck constant and $\mathcal{J}$ the rotational quantum number. \\
Energy(diferences) can be conveniently expressed in \textbf{wavenumbers} (Kaisers)
\begin{equation}
\tilde{\nu} = \frac{\Delta E}{hc} \qquad \left[\frac{1}{cm}\right]
\end{equation}
or in frequency (Hz)
\begin{equation}
\nu = \frac{\Delta E}{h} \qquad \left[Hz\right],
\end{equation}
so they translate directly to the observed spectrum.\\
The energy levels can be expressed in wavenumbers by
\begin{equation}
\epsilon_\mathcal{J} = \frac{E_\mathcal{J}}{hc} = \frac{h}{\delta \pi^2 I c} \mathcal{J}(\mathcal{J}+1).
\end{equation}
$\delta$ is Dirac's delta function and c the speed of light in vacuum. The fracture on the right side of the equation is referred to as the rotational constant B. It is inversely proportional to the moment of inertia $I$. From Quantum mechanics it turns out, that the only transitions that are allowed are those where $\Delta \mathcal{J} = \pm1$. Therefore we can only get energy levels $\epsilon_J$ that are 0, 2B, 6B, 12B, 20B, ..., all other transitions are forbidden, which results in equidistant spectral lines. \\
This is an example of the selection rule. So the full spectrum of a diatomic molecule is given by
\begin{equation}
\tilde{\nu}_{\mathcal{J}\rightarrow \mathcal{J}+1} = 2B(\mathcal{J}+1), \qquad \mathcal{J} = 0, 1, 2, ...\,.
\end{equation}






% CHAPTER 3 - VIBRATIONAL SPECTRA
\section{Vibrational spectra}
Now we will turn to vibration, another mode of molecular internal energy. Just like rotational energy, vibrational energy is quantified, hence can lead to spectroscopic transitions (= absorption lines). \\
We start off by looking at a harmonic oscillator in regular mechanics. Assume there is a restoring force,
\begin{equation}
f = -k (r-r_{eq}) \qquad (\text{Hooke's law})
\end{equation}
at the point $r$ from the equilibrium point $r_{eq}$ with a constant $k$, then the associated energy is
\begin{equation}
E = \frac{1}{2}k (r-r_{eq})^2 .
\end{equation}
In Quantum Mechanics energy levels turn out to be 
\begin{equation}
E_\nu = (\nu+\frac{1}{2})\hbar\tilde{\omega} \qquad (\nu = 0, 1, 2, ...).
\end{equation}
Transformed into spectroscopic units of Kaiser we get
\begin{equation}
\epsilon_\nu = \frac{E_\nu}{hc} = (\nu + \frac{1}{2})\tilde{\nu}.
\end{equation}
So what is $\omega$? If $f$ is really proportional to $\Delta r$, then it is a constant:
\begin{equation}
\tilde{\omega} = \sqrt{\frac{k}{\mu}}
\end{equation}
FIXME: What are the names of $k$ and $\mu$?\par
The lowest vibrational level is at
\begin{equation}
E_0 = \frac{1}{2} \hbar \tilde{\omega}.
\end{equation}
Therefore a molecule can never have \textbf{zero} vibrational energy! This is different from rotation, where the lowest state had zero energy. \par
FIXME: include slide 5 with mostly graphics. \par

The frequency of the observed line corresponds to the energy difference $\tilde{\nu}$ between adjacent levels (which corresponds to the classical frequency at which the molecule would vibrate. If molecules rotate, the Energy separation is 1 - 10\,cm$^{-1}$ whereas if they vibrate, the Energy separation is 100 - 10\,000\,cm$^{-1}$. This is a large difference in characteristic energy. We can therefore assume that the two motions occur independently.
 (This is a case of the Born- Oppenheimer approximation, which strictly also includes vibration. We write
 \begin{gather}
E_{total} = E_{rot} + E_{vib} \qquad \left[J\right]  \\
\epsilon_{total} = \epsilon_{rot} + \epsilon_{vib} \qquad \left[cm^{-1}\right].
 \end{gather}
Combining the discrete energy levels results in
\begin{equation}
\epsilon_{\mathcal{J},\nu} = \epsilon_{\mathcal{J}} + \epsilon_{\nu} = B \mathcal{J} (\mathcal{J} + 1) + (\nu + \frac{1}{2})\tilde{\nu}.
\end{equation}
Note: There are higher order therms in both rotation (zentrifugal stretching) and vibration (non-constant restoring force) that I am ignoring here. \par
 FIXME: insert figure from slide 9. \par
 The selection rule states, that
 \begin{equation}
\Delta \nu = \pm 1 \text \quad {and} \quad \Delta \mathcal{J} = \pm 1.
\end{equation}
This means, we normally cannot observe the vibrational fundamental directly. Rotational levels will be filled to varying degrees in a population of molecules. This results in different line intensities. \\
The transition frequencies are divided into two branches:
\begin{gather*}
\Delta \mathcal{J} = +1 \quad (\text{R-Branch}) \\
\qquad \Delta \epsilon_{\mathcal{J},\nu} = \tilde{\nu} + 2 B ( \mathcal{J''} + 1), 
	\qquad \mathcal{J''} = 0, 1, 2  \\
\Delta \mathcal{J} = -1 \quad (\text{P-Branch}) \\
\qquad \Delta \epsilon_{\mathcal{J},\nu} = \tilde{\nu} - 2 B ( \mathcal{J'} + 1) , 
	\qquad \mathcal{J'} = 0, 1, 2  \\ 
\end{gather*}
 They can be combined into: 
 \begin{equation}
\Delta \epsilon_{\mathcal{J},\nu} = \tilde{\nu} + 2 B m , 
	\qquad m = \pm1, \pm2, \pm3, ...
\end{equation}
These are the frequencies of the observed lines. \\
So why are the branches called 'P' and 'R'? For more complicated molecules, $\Delta \mathcal{J} = 0 \text{ and } \pm 2$ may also be allowed. All possible branches are:


\begin{center}
\begin{tabular}{|c|c|c|c|c|c|} \hline
$\Delta\mathcal{J}$ = & -2 & -1 & 0 & +1 & +2 \\
\hline
Name & O & P & Q & R & S \\
\hline
\end{tabular}
\end{center}

H$_2$O and CO$_2$ have three fundamental vibrations. These are:

\begin{center}
\begin{tabular}{| c | c | c | }
  \hline
  Symmetric stretch & Symmetric bend & Antisymmetric stretch  \\
  \hline
  % H2O
\schemestart
\chemfig{@{a1}H-[:30,0.8]@{a2}O-[:-30,0.8]@{a3}H}
\arrow(@a1.south west--.south west){->}[210,0.6,line width=.4pt,blue]
\arrow(@a2.north--.north){->}[90,0.6,line width=.4pt,blue]
\arrow(@a3.south east--.north east){->}[-30,0.6,line width=.4pt,blue]
\schemestop &
\schemestart
\chemfig{@{a1}H-[:30,0.8]@{a2}O-[:-30,0.8]@{a3}H}
\arrow(@a1.north--.north){->}[90,0.6,line width=.4pt,blue]
\arrow(@a2.south--.south){->}[-90,0.6,line width=.4pt,blue]
\arrow(@a3.north--.north){->}[90,0.6,line width=.4pt,blue]
\schemestop &
\schemestart
\chemfig{@{a1}H-[:30,0.8]@{a2}O-[:-30,0.8]@{a3}H}
\arrow(@a1.south west--.south west){->}[210,0.6,line width=.4pt,blue]
\arrow(@a2.north east--.east){->}[0,0.6,line width=.4pt,blue]
\arrow(@a3.north east--.north west){->}[145,0.6,line width=.4pt,blue]
\schemestop  \\
  $\tilde{\nu}_1 = 3651.7 \text{\,cm}^{-1}$ &
  $\tilde{\nu}_2 = 1595.0 \text{\,cm}^{-1}$ &
  $\tilde{\nu}_3 = 3755.8 \text{\,cm}^{-1}$ \\
  \hline
  %CO2
\schemestart
\chemfig{@{a1}O-[,0.8]@{a2}C-[,0.8]@{a3}O}
\arrow(@a1.west--.west){->}[180,0.6,line width=.4pt,blue]
\arrow(@a3.east--.east){->}[0,0.6,line width=.4pt,blue]
\schemestop &
\schemestart
\chemfig{@{a1}O-[,0.8]@{a2}C-[,0.8]@{a3}O}
\arrow(@a1.north--.north){->}[90,0.6,line width=.4pt,blue]
\arrow(@a2.south--.south){->}[270,0.6,line width=.4pt,blue]
\arrow(@a3.north--.south){->}[90,0.6,line width=.4pt,blue]
\schemestop &
\schemestart
\chemfig{@{a1}O-[,0.8]@{a2}C-[,0.8]@{a3}O}
\arrow(@a1.north west--.north){->}[0,0.6,line width=.4pt,blue]
\arrow(@a2.south east--.south){->}[180,0.6,line width=.4pt,blue]
\arrow(@a3.east--.east){->}[0,0.6,line width=.4pt,blue]
\schemestop \\
  $\tilde{\nu}_1 = 1330.0 \text{\,cm}^{-1}$ &
  $\tilde{\nu}_2 = 667.3 \text{\,cm}^{-1}$ &
  $\tilde{\nu}_3 = 2349.3 \text{\,cm}^{-1}$ \\
\hline
  
\end{tabular}
\end{center}




% CHAPTER 4 - LINE SHAPE
\section{Line shape}

% CHAPTER 5 - THERMAL RADIATION
\section{Thermal radiation}

FIXME: Mostly covered by script from Bachelor lecture, but at least
state the Planck law here.

% CHAPTER 6 - RADIATIVE TRANSFER EQUATION
\section{Radiative transfer equation}
The radiative transfer equation (RTE) describes the change of radiation along a path,
for example from the ground to a sensor above the surface. The RTE is expressed in 
terms of the monochromatic Intensity $I$. It can be written as
\begin{equation}
	\frac{dI}{ds} = - (\alpha + \sigma)I + \alpha B(T) + \sigma \int_\Omega PI 
	\frac{d\Omega}{4\pi}
\end{equation}
where $ds$ is a path element. The first term on the right hand side is 
the radiation loss due to extinction by absorption (absorption coefficient $\alpha$) 
and scattering (scattering coefficient $\sigma$). The second term is a source term
due to thermal emission $B(T)$. The third term is a source term due radiation that
is scattered into the path. It is calculated using the phase matrix $P$ of the 
scattering particles. The calculation involves an integration of the intensity over the 
whole solid angle $\Omega$. Note that normally, every variable depends on the frequency
and the viewing direction. \\
We see that if we include scattering in the RTE, it is very difficult to solve, because 
in order to compute the intensity in one direction, we need to know the intensity in 
every other direction.\\
One of the most used simplifications of the RTE is to neglect scattering. If we look at
thermal radiation, this is mostly true for a clear sky atmosphere. If we neglect 
scattering, the last term as well as the $\sigma$ in the first term vanish and the 
(clear sky) RTE looks like this:
\begin{equation}
	\frac{dI}{ds} = -\alpha I + \alpha B(T) = \alpha ( B(T) - I )
\end{equation}
This equation is also known as Schwarzschild's equation because if was formulated by 
K. Schwarzschild in 1906.\\
Using this equations, let's look at what happens in a homogeneous medium ($\alpha$ and $T$
constant). If the radiation passes through the medium long enough ($s \rightarrow \infty$)
the difference between $B$ and $I$ has to get smaller and smaller. This means that eventually,
$I$ will converge to $B$. This makes sense because if we look through a thick enough medium, 
we hardly see which radiation came into the medium, we only see the emission of the medium.\\
Schwarzschild's equation can be integrated along a path through a layer
(a complete derivation can be found in Petty, Eq. 8.5 - 8.13):
\begin{equation}
	I(s) = I(0) e^{-\tau (0,s)} + \int_{0}^{s} \alpha (s') B(s') 
	e^{-\tau (0,s')} ds' 
\end{equation}
where $\tau(0,s)$ is the opacity between the points. It is defined as:
\begin{equation}
	\tau (0,s) = \int_{0}^{s} \alpha(s') ds'.
\end{equation}
We see that the layer emits radiation, while the incoming radiation $I(0)$ is 
attenuated exponentially. \\
In radiative transfer models such as ARTS, the atmosphere is usually divided 
into homogeneous layers and the RTE is calculated through these layers. So to understand
the calculations in the model, it is useful to look at the solution of Schwarzschild's
equation in a homogeneous layer. The only difference is the opacity, which in a homogeneous
layer, we can write as
\begin{equation}
	\tau(0,s) = \int_{0}^{s} \alpha(s') ds' = \alpha s.
\end{equation}
To get the solution, we just plugin a homogeneous layer from 0 to $s$ into the integral 
form of the equation:
\begin{equation}
	I(s) = I(0) e^{-\alpha s} + \alpha B(T)\int_{0}^{s} e^{-\alpha (s-s')} ds' 
\end{equation}
which, after some algebra, gives this:
\begin{equation}
	I(s) = I(0) e^{-\alpha s} + B(T) \left( 1-e^{-\alpha s} \right).
\end{equation}
With the definition of the transmission $t = e^{-\alpha s}$, we can rewrite the solution as
\begin{equation}
	I(s) = t I(0)  + \left( 1-t \right) B(T).
\end{equation}
Let's look at a few extreme cases of the solution.\\
\underline{Optically thin:}\\
What happens if the opacity of the layer is very high, so $ t\rightarrow0$ ? 
Then we just see the Planck emission of the layer. the incoming radiation does not reach us.\\
\underline{Optically thin:}\\
What happens if the opacity of the layer is 0, to $ t \rightarrow \infty$ ? Then we just see
the incoming radiation $I(0)$. the layer doesn't do anything.\\
An interesting special case is when the opacity is small, but not zero. For small opacities, 
we can use a linear approximation of the transmission, which comes from the Taylor expansion
of the exponential function:
\begin{equation}
	t = e^{-\tau} \approx 1 - \tau.
\end{equation}
If we plug this into the solution of a homogeneous layer, we get
\begin{equation}
	I(s) = t I(0)  + \left( 1-t \right) B(T) \approx I(0) + \tau \left( B - I(0)\right)
\end{equation}
which has exactly the same form as Schwarzschild's equation itself.
%FIXME: Mostly covered by script from Bachelor lecture, but at least
%state the RTE here.

% CHAPTER 7 - JACOBIANS
\section{Jacobians}
 
\end{document}
