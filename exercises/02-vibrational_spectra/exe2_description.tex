%% LyX 2.3.3 created this file.  For more info, see http://www.lyx.org/.
%% Do not edit unless you really know what you are doing.
\documentclass[american]{article}
\usepackage{lmodern}
\usepackage[T1]{fontenc}
\usepackage[latin9]{inputenc}
\usepackage[a4paper]{geometry}
\geometry{verbose,tmargin=1cm,bmargin=4cm,lmargin=3cm,rmargin=3cm}
\usepackage{color}
\usepackage{amstext}
\usepackage[authoryear]{natbib}

\makeatletter
%%%%%%%%%%%%%%%%%%%%%%%%%%%%%% User specified LaTeX commands.
\usepackage[usenames,dvipsnames,svgnames,table]{xcolor}

\makeatother

\usepackage{babel}
\usepackage{listings}
\lstset{basicstyle={\footnotesize\bfseries\ttfamily},
breaklines=true,
commentstyle={\itshape\color{OliveGreen}},
keywordstyle={\bfseries\color{blue}},
showstringspaces=true,
stringstyle={\bfseries\color{RoyalPurple}}}
\renewcommand{\lstlistingname}{Listing}

\begin{document}

\title{Advanced radiation and remote sensing }

\author{Manfred Brath, Oliver Lemke, Stefan B�hler}
\maketitle

\subsection*{Exercise No. 2 -- Vibration}

You can reuse the Jupyter Notebook from the first exercise to answer
the questions on vibrational spectra. First, copy the Jupyter Notebook and
the python module from the last exercise into the directory of exercise no. 2:\\

\begin{lstlisting}
$ cd ~/arts-lectures/exercises/02-vibrational_spectra/ 
$ cp ../01-rotational_spectra/absorption.ipynb vibration.ipynb
$ cp ../01-rotational_spectra/absorption_module.py .

\end{lstlisting}
\medskip{}
Next, you have to adjust the frequency limits. For plotting in the
infrared range, it is common to use wavenumber in $\left[\text{cm}^{-1}\right]$
instead of frequency. Adapt the plotting part of your Jupyter notebook
accordingly. 
\begin{enumerate}
\item Find the fundamental band of $\text{CO}$ and plot its spectrum. 
\begin{itemize}
\item Determine the band center frequency $\hat{\nu}$ from your plot. 
\item There is some \textquotedblleft pollution\textquotedblright{} in the
P-branch that comes from lines of $^{13}\text{CO}$. Recalculate the
spectrum for the main isotopologue only by setting the species to
``CO-26''. What does the ``-26'' suffix mean? 
\end{itemize}
\item Explore the spectrum of either $\text{H}_{2}\text{O}$ or $\text{CO}_{2}$.
Can you find the different vibration bands? 
\end{enumerate}

\end{document}
