\documentclass[american]{article}
\usepackage{lmodern}
\usepackage[latin9]{inputenc}
\usepackage[a4paper]{geometry}
\geometry{verbose,tmargin=1cm,bmargin=4cm,lmargin=3cm,rmargin=3cm}
\usepackage{color}
\usepackage{array}
\usepackage{multirow}
\usepackage{amstext}
\usepackage[authoryear]{natbib}

\makeatletter

%%%%%%%%%%%%%%%%%%%%%%%%%%%%%% LyX specific LaTeX commands.
%% Because html converters don't know tabularnewline
\providecommand{\tabularnewline}{\\}

%%%%%%%%%%%%%%%%%%%%%%%%%%%%%% User specified LaTeX commands.
\usepackage[usenames,dvipsnames,svgnames,table]{xcolor}

\makeatother

\usepackage{babel}
\usepackage{listings}
\lstset{basicstyle={\footnotesize\bfseries\ttfamily},
breaklines=true,
commentstyle={\itshape\color{OliveGreen}},
keywordstyle={\bfseries\color{blue}},
showstringspaces=true,
stringstyle={\bfseries\color{RoyalPurple}}}
\begin{document}
\title{Advanced radiation and remote sensing }
\author{Manfred Brath, Oliver Lemke, Stefan B�hler}
\maketitle

\subsection*{Exercise No. 5 -- Jacobian and opacity rule}
\begin{enumerate}
\item Run the Jupyter Notebook jacobian.ipynb to calculate the brightness temperature
spectrum in nadir direction and the zenith opacity around the $183\,\text{GHz}$
line of water vapor for a midlatitude summer atmosphere. Answer following
question:
\begin{itemize}
\item Are there window regions?
\end{itemize}
The atmospheric temperature profile for the calculation was:\medskip{}

\centerline{%
\begin{tabular}{c|c|c}
\hline 
\multirow{2}{*}{\textbf{Pressure} $\left[\text{hPa}\right]$} & \multirow{2}{*}{\textbf{Temperature} $\left[\text{K}\right]$} & \multirow{2}{*}{\textbf{Altitude} $\left[\text{km}\right]$}\tabularnewline
 &  & \tabularnewline
\hline 
1013.0 & 294.2 & 0\tabularnewline
902.0 & 289.7 & 1\tabularnewline
802.0 & 285.2 & 2\tabularnewline
710.0 & 279.2 & 3\tabularnewline
628.0 & 273.2 & 4\tabularnewline
554.0 & 267.2 & 5\tabularnewline
487.0 & 261.2 & 6\tabularnewline
426.0 & 254.7 & 7\tabularnewline
372.0 & 248.2 & 8\tabularnewline
324.0 & 241.7 & 9\tabularnewline
281.0 & 235.3 & 10\tabularnewline
243.0 & 228.8 & 11\tabularnewline
209.0 & 222.3 & 12\tabularnewline
179.0 & 215.8 & 13\tabularnewline
153.0 & 215.7 & 14\tabularnewline
130.0 & 215.7 & 15\tabularnewline
111.0 & 215.7 & 16\tabularnewline
95.0 & 215.7 & 17\tabularnewline
81.2 & 216.8 & 18\tabularnewline
69.5 & 217.9 & 19\tabularnewline
59.5 & 219.2 & 20\tabularnewline
\hline 
\end{tabular}}\medskip{}

Consider the table and answer following questions:
\begin{itemize}
\item From which altitude does the radiation at the peak of the line ($\approx183\,\text{GHz}$)
originate? 
\item From which altitude does the radiation at the wing ($150\,\text{GHz}$)
originate? 
\end{itemize}
\item Change the variable $\texttt{highlight\_frequency}$ from $\texttt{None}$
to any desired frequency in $[\text{Hz}]$ within the range of the
brightness temperature spectrum of task 1 and rerun the last notebook cell. 
This will calculate the water vapor Jacobian and the
opacity $\tau$ between the top of the atmosphere $z_{TOA}$ and altitude
$z$ for the selected frequency. Additionally, a circle marks the
selected frequency in the plot of the brightness temperature spectrum
and in the plot of the zenith opacity.

Write down the altitude of the Jacobian peak and the altitude where
the opacity reaches 1 for some different frequencies and answer following
questions:
\begin{itemize}
\item Why are the altitude where the opacity reaches 1 and the altitude
of the Jacobian peak not exactly the same?
\item Why are the Jacobians sometimes positive and sometimes negative?
\end{itemize}
\end{enumerate}

\end{document}
