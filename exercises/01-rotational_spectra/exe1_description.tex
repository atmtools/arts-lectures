\documentclass[american]{article}
\usepackage{lmodern}
\usepackage[T1]{fontenc}
\usepackage[latin9]{inputenc}
\usepackage[a4paper]{geometry}
\geometry{verbose,tmargin=1cm,bmargin=4cm,lmargin=3cm,rmargin=3cm}
\usepackage{amstext}
\usepackage[authoryear]{natbib}
\usepackage{babel}
\begin{document}

\title{Advanced radiation and remote sensing }

\author{Manfred Brath, Oliver Lemke, Stefan B�hler}
\maketitle

\subsection*{Exercise No. 1 -- Calculation of absorption coefficients }
\begin{enumerate}
\item Calculate the absorption cross sections in the microwave spectral
range for the following molecules:
\begin{itemize}
\item $\text{HCl}$
\item $\text{ClO}$
\item $\text{CO}$
\item $\text{N}_{2}\text{O}$
\item $\text{O}_{3}$
\end{itemize}
Unless otherwise specified use the parameter setting as given in the
Jupyter (IPython) notebook "absorption.ipynb"\,.\\
Questions and tasks:
\begin{enumerate}
\item Estimate the rotational constant $B$ (in $\text{GHz}$) for $\text{HCl}$
and for $\text{CO}$. 
\item Why is $B$ larger for $\text{HCl}$ than for $\text{CO}$? 
\item Do you have any idea why $\text{N}_{2}\text{O}$ behaves like a diatomic
molecule -- and $\text{O}_{3}$ not?
\item Calculate the reduced mass (in atomic mass units $\text{u}$) of the
different molecules from the masses of the individual atoms. (For
the diatomic molecules this is trivial. For $\text{N}_{2}\text{O}$,
The appropriate mass can be found by careful thinking. Ignore
$\text{O}_{3}$.) 
\item Calculate the bond length (in pm) of the various molecules (except
$\text{O}_{3}$) from the reduced mass and the rotational constant.
Verify your result with Google. Again $\text{N}_{2}\text{O}$ needs
some extra thinking. 
\item Play with different temperatures. How does the rotational spectrum
change? Can you explain the changes? 
\end{enumerate}
\item Investigate some other molecules! 
\item Show for a diatomic molecule that the moment of inertia is given by
\begin{equation}
I=\mu r_{0}^{2}
\end{equation}
 where $\mu$ is the reduced mass, defined as
\begin{equation}
\mu=\frac{m_{1}m_{2}}{m_{1}+m_{2}}
\end{equation}
\end{enumerate}

\end{document}
