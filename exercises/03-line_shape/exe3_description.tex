\documentclass[american]{article}
\usepackage{lmodern}
\usepackage[latin9]{inputenc}
\usepackage[a4paper]{geometry}
\geometry{verbose,tmargin=1cm,bmargin=4cm,lmargin=3cm,rmargin=3cm}
\usepackage{color}
\usepackage[authoryear]{natbib}

\makeatletter
%%%%%%%%%%%%%%%%%%%%%%%%%%%%%% User specified LaTeX commands.
\usepackage[usenames,dvipsnames,svgnames,table]{xcolor}

\makeatother

\usepackage{babel}
\usepackage{listings}
\lstset{basicstyle={\footnotesize\bfseries\ttfamily},
breaklines=true,
commentstyle={\itshape\color{OliveGreen}},
keywordstyle={\bfseries\color{blue}},
showstringspaces=true,
stringstyle={\bfseries\color{RoyalPurple}}}
\renewcommand{\lstlistingname}{Listing}

\begin{document}
\title{Advanced radiation and remote sensing }
\author{Manfred Brath, Oliver Lemke, Stefan B�hler}
\maketitle

\subsection*{Exercise No. 3 -- Line shape}

You can reuse the python script from the first exercise as a starting
point for this exercise. First, copy the python script from the first
exercise into the directory of exercise no. 3:

\begin{lstlisting}
$ cd ~/arts-lectures/exercises/03-line_shape/ 
$ cp ../01-rotational_spectra/absorption.ipynb line_shape.ipynb
$ cp ../01-rotational_spectra/absorption_module.py .
\end{lstlisting}
\medskip{}
The ``calculate\_absxsec'' function within your ipython notebook calculates
absorption cross sections and uses several keyword arguments as inputs,
among other lineshape and normalization, see also the function definition 
in ``absorption\_module.py" or the contextual help. 
Please use ``VP'' (Voigt-Function) as line shape and ``VVH''  (van Vleck-Huber) as normalization:\\

\begin{enumerate}
\item Choose an individual line of for example water vapor and perform calculations
over a restricted frequency range for a number of different pressures.
Keep the temperature constant.
\begin{itemize}
\item How does the shape of the spectral lines change?
\end{itemize}
By now we investigated absorption in terms of the absorption cross-section
$\sigma$. Another widely used unit is the absorption coeffiction
$\alpha$. It takes the number concentration $n$ of the absorber
into account:
\begin{equation}
\alpha=n\cdot\sigma
\end{equation}

\begin{itemize}
\item How does the absorption coefficient in the line centre change, if
pressure is changed?
\end{itemize}
\item The full-width at half maximum (FWHM) is a measure of the line width.
Typhon provides the function \emph{typhon.spectroscopy.linewidth()}
to calculate the FWHM for a given absorption spectrum. 
\begin{itemize}
\item Make a plot of this as a function of altitude (pressure) for a microwave
line and an infrared absorption line.
\end{itemize}
\end{enumerate}

\end{document}
