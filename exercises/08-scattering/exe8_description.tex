%% LyX 2.3.3 created this file.  For more info, see http://www.lyx.org/.
%% Do not edit unless you really know what you are doing.
\documentclass[american]{article}
\usepackage{lmodern}
\usepackage[latin9]{inputenc}
\usepackage[a4paper]{geometry}
\geometry{verbose,tmargin=1cm,bmargin=4cm,lmargin=3cm,rmargin=3cm}
\usepackage{color}
\usepackage{amstext}
\usepackage[authoryear]{natbib}

\makeatletter
%%%%%%%%%%%%%%%%%%%%%%%%%%%%%% User specified LaTeX commands.
\usepackage[usenames,dvipsnames,svgnames,table]{xcolor}

\makeatother

\usepackage{babel}
\usepackage{listings}
\lstset{basicstyle={\footnotesize\bfseries\ttfamily},
breaklines=true,
commentstyle={\itshape\color{OliveGreen}},
keywordstyle={\bfseries\color{blue}},
showstringspaces=true,
stringstyle={\bfseries\color{RoyalPurple}}}
\begin{document}
\title{Advanced radiation and remote sensing }
\author{Manfred Brath, Oliver Lemke, Stefan B�hler}
\maketitle

\subsection*{Exercise No. 8 -- Scattering}
\begin{enumerate}
\item Run the Jupyter notebook \texttt{scattering.ipynb}.
This will simulate the radiation field at a frequency of $229\,\text{GHz}$ 
for an atmosphere with an ice cloud as well for clear-sky. Since this 
is a one-dimensional simulation (vertical dimension only), the calculated 
radiation fields have two dimensions: altitude (pressure) and zenith angle. 
The  Jupyter notebook will plot the two radiation fields in the atmosphere at 
a zenith angle of $180{^\circ}$. Here, the zenith angle describes the viewing
direction. This means that you are looking at the upward directed radiation. 
The unit is brightness temperature. 
\begin{enumerate}
\item Describe the difference between cloudy and clear-sky radiation. 
\item Guess where the ice cloud is located in the atmosphere based on the
two radiation fields? 
\item Explain the difference between cloudy and clear-sky radiation.
\end{enumerate}
\item Change the zenith angle (\texttt{zenith\_angle}) from $180{^\circ}$
to $0{^\circ}$ and rerun the corresponding cell within the notebook.
\begin{enumerate}
\item Describe and explain the difference.
\item Why is the brightness temperature at the top of the atmosphere so
low?
\end{enumerate}
\item Now you will look at the radiation fields as a function of zenith
angle (viewing direction) at a fixed pressure level. In the Jupyter
notebook, change the variable \texttt{pressure\_level} from \texttt{None}
to a pressure level in $\left[\text{Pa}\right]$, which is within
the ice cloud and rerun the corresponding cell within the notebook. 
\begin{enumerate}
\item Explain the shape of the radiation field without the cloud. 
\item How does the radiation field with the cloud differ? 
\end{enumerate}
\item Make the same calculation as in task 3 but with a less or a more dense
ice cloud. To do that, you have to call the function \texttt{scattering()}
within your script with the argument \texttt{ice\_water\_path} set
to your desired value in $\left[\text{kg}\,\text{m}^{-2}\right]$.
The ice water path is the vertically integrated mass content of ice.
In task 3, the function \texttt{scattering()} used a default value
of $2\,\text{kg}\,\text{m}^{-2}$. 
\end{enumerate}

\end{document}
