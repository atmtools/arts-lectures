%% LyX 2.3.3 created this file.  For more info, see http://www.lyx.org/.
%% Do not edit unless you really know what you are doing.
\documentclass[american]{article}
\usepackage{lmodern}
\usepackage[latin9]{inputenc}
\usepackage[a4paper]{geometry}
\geometry{verbose,tmargin=1cm,bmargin=4cm,lmargin=3cm,rmargin=3cm}
\usepackage{color}
\usepackage{amstext}
\usepackage[authoryear]{natbib}

\makeatletter
%%%%%%%%%%%%%%%%%%%%%%%%%%%%%% User specified LaTeX commands.
\usepackage[usenames,dvipsnames,svgnames,table]{xcolor}

\makeatother

\usepackage{babel}
\usepackage{listings}
\lstset{basicstyle={\footnotesize\bfseries\ttfamily},
breaklines=true,
commentstyle={\itshape\color{OliveGreen}},
keywordstyle={\bfseries\color{blue}},
showstringspaces=true,
stringstyle={\bfseries\color{RoyalPurple}}}
\begin{document}
\title{Advanced radiation and remote sensing }
\author{Manfred Brath, Oliver Lemke, Stefan B�hler}
\maketitle

\subsection*{Exercise No. 7 -- Outgoing Longwave Radiation (OLR)}
\begin{enumerate}
\item Run ARTS on the Jupyter notebook olr.ipynb . This will calculate the spectrum
of outgoing longwave radiation for a midlatitude-summer atmosphere.
Our calculation trades accuracy for computational efficiency. For
example, we use only water vapor and carbon dioxide as absorbers.
We use only 300 frequency grid points and approximately 54,000 spectral
lines, whereas for accurate calculations one needs at least 10,000
frequency grid points and 500,000 spectral lines, taking into account
absorbing species like ozone and methane. The script plots the spectral
irradiance, in SI units, as a function of wavenumber. Planck curves
for different temperatures are shown for comparison. We integrate
the whole spectrum to quantify the power per square that is emitted
by the atmosphere (value in title).
\begin{enumerate}
\item How would the OLR spectrum look in units of brightness temperature? 
\item How would the Planck curves look in units of brightness temperature?
\item Find the $\text{CO}_{2}$ absorption band and the regions of $\text{H}_{2}\text{O}$
absorption. From which height in the atmosphere does the radiation
in the $\text{CO}_{2}$ band originate?
\item Are there window regions? 
\item What will determine the OLR in the window regions?
\item Use the plot to explain the atmospheric greenhouse effect.
\end{enumerate}
\item Investigate how the OLR changes for different atmospheric conditions
by modifying the input data. Use \texttt{atmfield.scale(...)}
and \texttt{atmfield.set(...)} to change the atmospheric data:
\begin{itemize}
\item Add $1\,$$\,\text{K}$ to the temperature.
\item Increase the $\text{CO}_{2}$ conentration by a factor of $2$.
\item Increase the $\text{H}_{2}\text{O}$ conentration by a factor of $1.2$.
\end{itemize}
\begin{enumerate}
\item Change it, and calculate and plot the spectrum for each change. 
\item Compare the spectra for humidity and carbon dioxide changes. Where
do the changes occur?
\item Compare the OLR numbers, which is the more potent greenhouse gas,
$\text{CO}_{2}$ or $\text{H}_{2}\text{O}$?
\item What is the effect of the temperature change? Explain.
\end{enumerate}
\end{enumerate}

\end{document}
