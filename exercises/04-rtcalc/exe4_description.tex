%% LyX 2.3.3 created this file.  For more info, see http://www.lyx.org/.
%% Do not edit unless you really know what you are doing.
\documentclass[american]{article}
\usepackage{lmodern}
\usepackage[latin9]{inputenc}
\usepackage[a4paper]{geometry}
\geometry{verbose,tmargin=1cm,bmargin=4cm,lmargin=3cm,rmargin=3cm}
\usepackage{color}
\usepackage{amstext}
\usepackage[authoryear]{natbib}

\makeatletter
%%%%%%%%%%%%%%%%%%%%%%%%%%%%%% User specified LaTeX commands.
\usepackage[usenames,dvipsnames,svgnames,table]{xcolor}

\makeatother

\usepackage{babel}
\usepackage{listings}
\lstset{basicstyle={\footnotesize\bfseries\ttfamily},
breaklines=true,
commentstyle={\itshape\color{OliveGreen}},
keywordstyle={\bfseries\color{blue}},
showstringspaces=true,
stringstyle={\bfseries\color{RoyalPurple}}}
\begin{document}
\title{Advanced radiation and remote sensing }
\author{Lukas Kluft, Manfred Brath, Stefan B�hler}
\maketitle

\subsection*{Exercise No. 4 -- Atmospheric Brightness Temperature Spectra}
\begin{enumerate}
\item Run the Python script rtcalc.py to calculate the spectrum of atmospheric
zenith opacity in the microwave spectral range for a midlatitude-summer
atmosphere over a wet land surface. 

Consider the zenith opacity spectrum to answer the following questions:
\begin{itemize}
\item The spectrum includes four spectral lines. To which species do these
lines belong? Play around with the absorption species selection in
the Python script.
\item We speak of window regions where the zenith opacity is below 1. Where
are they?
\end{itemize}
\item Brightness temperature is a unit for intensity. It is the temperature
that a blackbody should have to give the same intensity as measured.
Mathematically, the transformation between intensity in SI units and
intensity in brightness temperature is done with the Planck formula.
ARTS is capable to perform simulation in units of brightness temperature.
Uncomment the code part for the second task. Investigate the brightness
temperature spectra for different hypothetical sensors:
\begin{enumerate}
\item A ground-based sensor looking in the zenith direction.
\item A sensor on an airplane ($z=10\,\text{km}$) looking in the zenith
direction.
\end{enumerate}
Consider both opacity and brightness temperatures to answer the following
questions: 
\begin{itemize}
\item In plot (a), why do the lines near 60 GHz and near 180 GHz appear
flat on top? 
\item In plot (b), why is the line at 180 GHz smaller than before? 
\item Describe the difference between plots (a) and (b). What happens to
the lines, what happens to the background? Can you explain what you
see? 
\end{itemize}
\item Make the same calculation as in task 2 for a satellite sensor ($z=800\,\text{km}$)
looking nadir (straight down).

Answer following questions: 
\begin{itemize}
\item Explain the brightness temperature simulated in the window regions. 
\item Why does the line at 22 GHz look different from the others? 
\item Investigate the the $\text{O}_{2}$ line at $120\,\text{GHz}$. Perform
an ARTS simulation focused around that frequency. Why does the shape
close to the center of the $\text{O}_{2}$ line at $120\,\text{GHz}$
looks so differently compared to the $183\,\text{GHz}$. 
\end{itemize}
\end{enumerate}

\end{document}
