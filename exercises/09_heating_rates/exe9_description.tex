\documentclass[american]{article}
\usepackage{lmodern}
\usepackage[latin9]{inputenc}
\usepackage[a4paper]{geometry}
\geometry{verbose,tmargin=1cm,bmargin=4cm,lmargin=3cm,rmargin=3cm}
\usepackage{color}
\usepackage{amstext}
\usepackage[authoryear]{natbib}

\makeatletter
%%%%%%%%%%%%%%%%%%%%%%%%%%%%%% User specified LaTeX commands.
\usepackage[usenames,dvipsnames,svgnames,table]{xcolor}

\makeatother

\usepackage{babel}
\usepackage{listings}
\lstset{basicstyle={\footnotesize\bfseries\ttfamily},
%	breaklines=true,
commentstyle={\itshape\color{OliveGreen}},
keywordstyle={\bfseries\color{blue}},
showstringspaces=true,
stringstyle={\bfseries\color{RoyalPurple}}}
\begin{document}
\title{Advanced radiation and remote sensing }
\author{Manfred Brath, Oliver Lemke, Stefan B�hler}
\maketitle

\subsection*{Exercise No. 9 -- Heating rate}

The heating rate denotes the change of atmospheric temperature with time due
to gain or loss of energy. Here, we consider only the gain or loss due to
radiation.
The heating rate including only radiation is 
\begin{equation}
\frac{\partial T\left(z\right)}{\partial t}=-\frac{1}{\rho\left( z \right) c_p }\frac{\partial}{\partial z}F_{net}\left( z \right)
\end{equation}
with $ \rho\left( z \right)$ the density of dry air\footnote{To keep it simple, we assume dry air. 
In reality the air is not dry. Nonetheless, the differnces are small.}, $ c_p = 1.0035\, \text{J}\, \text{kg}^{-1} \text{K}^{-1}$ the specific heat 
capacity of dry air and $ F_{net}$ the net radiation flux. 
The net radiation flux is 
\begin{equation}
F_{net}=F_{up}-F_{down}
\end{equation}
with $F_{up}$ and $F_{down}$ the up- and downward radiation flux (irradiance), respectively. 
The density of dry air is 
\begin{equation}
\rho  =\frac{p}{R_s\,T}
\end{equation}
with pressure $ p $, temperature  $ T $ and the specific gas constant 
$R_s = 287.058\, \text{J}\,\text{kg}^{-1} \text{K}^{-1}$.

\begin{enumerate}
\item Run the Jupyter notebook heating\_rates.ipynb. This will calculate 
the upward and downward radiation fluxes. Calculate the 
net flux and plot upward, downward and net flux together in one figure against altitude.
Explain the plot.
\item Implement the function \texttt{calc\_heatingrates(...)}. Use the function to calculate 
the heating rate. Plot the heating rate against altitude and explain the plot. How would a 
heating rate look like in thermal equilibrium?

\item Uncomment the line with the function call \texttt{calc\_spectral\_irradiance} and
calculate the spectral upward, downward and net flux. 

\item Use the function \texttt{integrate\_spectral\_irradiance(...)} to integrate the spectral
irradiance over three continuing bands:
\begin{enumerate}
\item the far infrared
\item the $ \text{CO}_2 $-band
\item the window-region and above.
\end{enumerate}
Calculate the heating rate for each band and plot them together with the total heating rate from Task 2.
Compare the band heating rates with the total heating rate and explain differences.
\end{enumerate}
	
\end{document}
